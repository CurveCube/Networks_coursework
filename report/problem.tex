В последнее время активно развиваются проекты спутникового интернета (Starlink, OneWeb, Сфера, Бюро 1440), в основе которых лежат низкорбитальные группировки. Предлагается смоделировать связь двух наземных терминалов через спутниковую группировку. Особенностью подобных группировок является использование низких орбит, что приводит к тому, что в зоне видимости терминалов постоянно находятся разные спутники, так же это осложняет связь между самими спутниками. Это делается для большего покрытия (связь со спутником на геостационарной орбите невозможна для терминала выше восьмидесятой параллели) и для меньших задержек. Ставится задача моделирования связи двух терминалов через такую группировку.