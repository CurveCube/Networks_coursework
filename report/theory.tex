Для моделирования движения спутников используется простейшая модель, не учитывающая торможение и другие возмущения. По Кеплеровским элементам орбиты (большая полуось, эксцентриситет, наклонение, долгота восходящего узла, аргумент перицентра, средняя аномалия) в момент $t_0$ предсказывается положение спутника в момент времени $t$.

Наличие связи между спутниками определяется по предельному углу (чтобы Земля не перекрывала линию прямой видимости). Связь между спутником и терминалом так же определяется по углу (ограниченному, чтобы избежать связи со спутником слишком близким к горизонту).

Для нахождения оптимального пути между терминалами используется протокол Open Shortest Path First, учитывающий расстояние между элементами сети. Для симуляции передачи данных используется протокол Selective Repeat. 

